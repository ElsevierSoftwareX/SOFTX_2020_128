\section{Introduction} \label{sec:introduction}
In 2016, the LIGO and Virgo collaborations announced the first detection of gravitational waves~\cite{Abbott:2016blz}. The signal, detected by the LIGO interferometers~\cite{Abramovici:1992ah}, is associated to a system of two black holes spiraling down and finally merging into a single and final black hole. This event is of very short duration, $\sim200$~ms, and had to be extracted from a very large population of transient noise events, also called glitches, polluting the data of interferometric detectors. Analysis tools have been specifically developed to search for rare gravitational-wave events buried in the detectors' noise. Even these tools show good intrinsic performance, they are critically limited by the presence of glitches which present features similar to true gravitational-wave signals. As a result, much efforts have been developed to understand, track and veto glitches~\cite{Aasi:2012wd,Aasi:2014mqd,TheLIGOScientific:2016zmo} in order to improve the sensitivity of searches.

Understanding glitches relies on a complete monitoring of the instrument and its environment. For that purpose, thousands of auxiliary data streams are recorded. They include signals from enviromnental sensors (thermic, acoustic, seismic, magnetic...), from photo-diodes, from actuators, from electronic devices or from feed-back control loops. These auxiliary channels are insensitive to gravitational waves and are used to witness disturbances from a noise source. In particular, transient events must be searched for in thousands of channels data in order to identify correlations with the detector's output data and understand coupling mechanisms leading to glitches. For that purpose, the Omicron algorithm was developed and used to detect transient events in auxiliary data of gravitational-wave detectors~\cite{Nuttall:2015dqa,TheLIGOScientific:2016zmo}. Events detected by Omicron are given a set parameters such as timing, frequency or signal strength, defining a trigger. This additional information is often useful in the process of understanding the glitch origin. Omicron triggers are processed by many other analysis tools and applications~\cite{Isogai:2010zz,Smith:2011an,gspy} to isolate coupling mechanisms or to classify glitches into families.

The Omicron algorithm is optimized to process many channels and to produce triggers promptly with limited computing resources. As a result, hundreds of LIGO and Virgo channels are processed by Omicron in quasi-real time. Therefore, Omicron triggers provides low-latency information about the data quality to the detector operators and to gravitational-wave searches running online.


Omicron is an algorithm designed to search for power excess in data collected by gravitational-wave detectors. Originally, a search tool called $Q$ pipeline~\cite{Chatterji:2004} was developed to search for astrophysically unmodeled bursts of gravitational radiation in data from networks of interferometric detectors~\cite{Abbott:2009zi,Abbott:2009up,Abadie:2010mt}.

The method is based on a multiresolution time-frequency transform that efficiently targets waveforms localized in time and frequency. 

\begin{itemize}
\item motivate the multi-resolution search
\item motivate good time-frequency resolution
\item introduce the big-dog event as an illustrative example
\end{itemize}

