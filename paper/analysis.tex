\section{Data analysis} \label{sec:analysis}


%%%%%%%%%%%%%%%%%%%%%%%%%%%%%%%%%%%%%%%%%%%%%%%%%%%%%%%%%%%%%%%%%%%%%%%%%%%%%%%
\subsection{The input data} \label{sec:analysis:data}

The Omicron algorithm is designed to process any data time series $x(t)$ associated to a gravitational-wave detector channel, e.g. the detector output channel, an environmental sensor, a photodiode signal, etc. The data consists of a discrete time sequence, $x[j]$, which is identified by a channel name and a native sampling frequency, $f_s$, defining the time separation between two consecutive data points $\delta t = t_{j+1}-t_j = 1/f_s$. As we will see in Sec.~\ref{sec:algorithm:conditioning}, the Omicron algorithm includes a data conditioning step where the input data is downsampled to a working sampling frequency $f_w$, allowing for an optimized subsequent processing. The data is then iteratively analyzed by time chunks of fixed duration $T_c$, corresponding to $N=T_cf_w$ data samples. With Omicron, the working sampling frequency $f_w$ and the chunk duration $T_c$ are required to each be a power of two. Therefore $N$ is a power of two in order to take advantage of the computational efficiency of the fast Fourier transform.

The Omicron analysis makes use of Fourier transforms~\cite{Gabor:1946} to decompose the signal into characteristic frequencies and to construct spectra. The discrete forward and inverse normalized Fourier transforms are written as
\begin{equation}
  \tilde{x}[k]=\frac{1}{f_w}\sum_{j=0}^{N-1}{x[j]\mathrm{e}^{-2i\pi jk/N}}
\end{equation}
and
\begin{equation}
  x[j]=\frac{f_w}{N}\sum_{j=0}^{N-1}{\tilde{x}[k]\mathrm{e}^{+2i\pi jk/N}}.
\end{equation}
The input data is a real time series whereas the frequency-domain data vector is complex-valued. As a result, the spectrum of the data is symmetrical around DC. The negative frequencies are ignored and only the one-sided data vector is considered. By convention, the first element, $\tilde{x}[0]$ is the DC component and the next elements, $\tilde{x}[k]$, $1\le k \le N/2$, are the spectrum coefficients for positive frequencies, $kf_w/N$. The last element, $\tilde{x}[N/2]$, is associated to the Nyquist frequency $f_w/2$. 

From these definitions, the one-sided periodogram of $x$ is computed as:
\begin{equation}
  P[k]= \frac{2}{T_c}\left| \frac{1}{f_w}\sum_{j=0}^{N-1}{x[j]\mathrm{e}^{-2i\pi jk/N}} \right|^2,
\end{equation}
where the factor 2 accounts for both negative and positive frequencies. When averaging multiple periodograms, one gets an estimate of the signal power spectral density (PSD), $S[k]$. Sec.~\ref{sec:analysis:whitening} describes in detail how the PSD is estimated. 

Before being applied the $Q$ transform, the data chunk is whitened such that individual samples of the discrete time sequence are statistically independent random variables. To achieve this, a simple method consists of dividing the 
\begin{equation}
  \tilde{x}_{white}[k] = \tilde{x}[k]/S_n[k]
\end{equation}


%%%%%%%%%%%%%%%%%%%%%%%%%%%%%%%%%%%%%%%%%%%%%%%%%%%%%%%%%%%%%%%%%%%%%%%%%%%%%%%%%%%%%%%%
\subsection{The $Q$ transform} \label{sec:analysis:qtransform}
To conduct the multi-resolution analysis motivated in the introduction, the data are processed using the $Q$ transform~\cite{Brown:1991}. The $Q$ transform is a modification of the standard short time Fourier transform in which the analysis window duration varies inversely with frequency such that the time-frequency plane is covered by tiles of constant quality factor $Q$. The signal time series, $x(t)$, is projected onto a basis of windowed sine waves:
\begin{equation}
  X(\tau, \phi, Q) = \int_{-\infty}^{+\infty}{ x(t) w(t-\tau,\phi,Q) e^{-2i\pi\phi t}dt}.
  \label{eq:qtransform1}
\end{equation}
The transformed coefficient, $X$, measures the average signal amplitude and phase within a time-frequency region centered on time $\tau$ and frequency $\phi$, whose shape and area are determined by the requested quality factor $Q$ and the particular choice of analysis window, $w$. This region is called a \textit{tile}.

To optimize the algorithmic implementation of the $Q$ transform, we use an alternative form of Eq.~\ref{eq:qtransform1}, which can be formulated in the frequency domain as~\cite{Chatterji:2004}:
\begin{equation}
  X(\tau, \phi, Q) = \int_{-\infty}^{+\infty}{ \tilde{x}(f+\phi) \tilde{w}^{*}(f,\phi,Q) e^{+2i\pi f \tau}df},
  \label{eq:qtransform2}
\end{equation}
where tilded variables represent Fourier-transformed quantities. When working with discrete data vectors of size $N$\footnote{All data sequences are assumed to be infinite and of period $N$}, Eq.~\ref{eq:qtransform2} becomes:
\begin{equation}
  X(m,l,q) = \frac{f_w}{N}\sum_{k=0}^{N-1}{\tilde{x}[k+l]\tilde{w}^{*}[k,l,q]e^{+2i\pi mk/N}},
  \label{eq:qtransform3}
\end{equation}
where $m$, $l$ and $q$ index the tiles in the three dimensions of time, frequency and quality factor respectively. In Eq.~\ref{eq:qtransform3}, the signal time series is applied a Fourier transform, a shift in frequency, a multiplication by the frequency-domain window and an inverse Fourier transform. This expression is favored when it comes to implement an efficient discrete $Q$ transform, as with the Omicron algorithm. Indeed, the input time series $x$ is only Fourier-transformed once and used as such for all the tiles. Moreover, the inverse Fourier transform is only performed for the frequencies and quality factors that we are interested in.

%%%%%%%%%%%%%%%%%%%%%%%%%%%%%%%%%%%%%%%%%%%%%%%%%%%%%%%%%%%%%%%%%%%%%%%%%%%%%%%%%%%%%%%%
\subsection{The Gaussian window} \label{sec:analysis:window}

\bluenote{Explain the best-case scenario of a Gaussian window}. Using a Gaussian window is a natural choice to provide a good time-frequency resolution. However, such a window has an infinite extent which makes it difficult to use in a discrete analysis framework. Instead, the Omicron algorithm approximates the Gaussian window with a bisquare window which has a simple form in the frequency domain:
\begin{equation}
  \tilde{w}(f,\phi,Q) = \tilde{w}^*(f,\phi,Q) =
  \begin{cases}
    W\left[1 - \left(f/\Delta f(\phi,Q)\right)^2 \right]^2 & |f| < \Delta f(\phi,Q), \\
    0 & \textrm{otherwise},
  \end{cases}
  \label{eq:bisquare}
\end{equation}
where $W$ is a normalization factor and $\Delta f(\phi,Q)=\phi\sqrt{11}/Q$ is the half bandwidth of the window. There are many advantages in using such a window~\cite{Chatterji:2004}. First, it is zero for $|f| > \Delta f(\phi,Q)$, so that shorter inverse Fourier transforms are performed to compute $\tilde{w}$ in Eq.~\ref{eq:qtransform3}. Second, it achieves a time-frequency localization that is only 4.5~\% greater than the minimum possible time-frequency uncertainty associated with the ideal Gaussian window. Third, it offers a limited energy leakage into time-domain side lobes. Fourth, thanks to the finite extent of the the bisquare window, it is possible to add protections against signal aliasing. Indeed, when using the two conditions
\begin{align}
  &Q\ge\sqrt{11} \qquad \text{and}\label{eq:antialias1} \\
  &\phi \le \frac{f_{\text{Nyquist}}}{1+\sqrt{11}/Q}, \label{eq:antialias2}
\end{align}
the zero frequency and the Nyquist frequency, $f_{\text{Nyquist}}$, are never exceeded.

In a discrete analysis, the bisquare window writes:
\begin{equation}
  \tilde{w}[k,l,q] = W \left[1 - \left(\frac{2k}{M[l,q]-1}-1\right)^2 \right]^2, \qquad 0\le k < M[l,q],
  \label{eq:dbisquare}
\end{equation}
where $M[l,q] = 2\lfloor{T_c\Delta f(\phi,Q)\rfloor} +1$ is the number of points where the bisquare function is non zero.

In the following, we will often refer to a basis of complex-valued sinusoidal Gaussian waveforms although bisquare windows are actually used for the Omicron implementation.


%%%%%%%%%%%%%%%%%%%%%%%%%%%%%%%%%%%%%%%%%%%%%%%%%%%%%%%%%%%%%%%%%%%%%%%%%%%%%%%%%%%%%%%%
\subsection{The tiles} \label{sec:analysis:tiles}
The tiles, indexed by $(m,l,q)$, must be chosen to cover a finite region of parameter space, $[\tau_{min};\tau_{max}]\times [\phi_{min};\phi_{max}] \times [Q_{min};Q_{max}]$. The density of tiles must be large to guarantee a high detection efficiency. However, the number of tiles must also be as small as possible to provide a fast computing processing. To meet these two competitive requirements, the parameter space is tiled such that the fractional energy loss due to mismatch is below a pre-defined value, $\mu_{max}$. For complex-valued sinusoidal Gaussian waveforms, the mismatch can be analytically computed and the optimal number of tiles can be determined. The Omicron algorithm adopts the tiling strategy developed in~\cite{Chatterji:2004}, where the tiles are distributed over a cubic lattice using a mismatch metric\footnote{The metric includes a non-diagonal $\delta \phi \delta Q$ term which has been neglected.} to measure distances:
\begin{equation}
  \delta s^2 =
  \frac{4\pi^2\phi^2}{Q^2}\delta \tau^2
  + \frac{2+Q^2}{4\phi^2}\delta \phi^2
  + \frac{1}{2Q^2}\delta Q^2.
  \label{eq:tilemetric}
\end{equation}
To guarantee a fractional energy loss smaller than $\mu_{max}$, the minimum number of tiles, $N_\tau \times N_\phi \times N_Q$, is given by the mismatch distances integrated over the three dimensions:
\begin{align}
  N_\tau \ge \frac{s_\tau}{2\sqrt{\mu_{max}/3}},  & \qquad s_\tau = \frac{2\pi\phi}{Q}(\tau_{max} - \tau_{min}), \label{eq:tiledistancetau} \\
  N_\phi \ge \frac{s_\phi}{2\sqrt{\mu_{max}/3}},  & \qquad s_\phi = \frac{\sqrt{2+Q^2}}{2}\ln(\phi_{max}/\phi_{min}), \label{eq:tiledistancephi} \\
  N_Q \ge \frac{s_Q}{2\sqrt{\mu_{max}/3}},  & \qquad s_Q = \frac{1}{\sqrt{2}}\ln(Q_{max}/Q_{min}). \label{eq:tiledistanceq}
\end{align}
This Omicron tiling structure can be depicted as a set of $N_Q$ logarithmically-spaced $Q$ planes:
\begin{equation}
  Q_q = Q_{min}\left[ \frac{Q_{max}}{Q_{min}}\right]^{(0.5+q)/N_q}, \qquad 0\le q < N_Q.
  \label{eq:q}
\end{equation}
Each of these planes is divided into $N_\phi$ logarithmically-spaced frequency rows:
\begin{equation}
  \phi_l = \phi_{min}\left[ \frac{\phi_{max}}{\phi_{min}}\right]^{(0.5+l)/N_\phi}, \qquad 0\le l < N_\phi.
  \label{eq:phi}
\end{equation}
A characteristic bandwidth is defined as the row width:
\begin{equation}
  \Delta\phi_l = \phi_{l} \left[ \left(\frac{\phi_{max}}{\phi_{min}}\right)^{0.5/N_\phi} - \left(\frac{\phi_{max}}{\phi_{min}}\right)^{-0.5/N_\phi} \right].
  \label{eq:dphi}
\end{equation}
Each frequency row is finally sub-divided into $N_\tau$ linearly-spaced time bins:
\begin{equation}
  \tau_m = -\frac{T_c}{2}+(m+0.5)\frac{T_c}{N_\tau}, \qquad 0\le m < N_\tau.
  \label{eq:tau}
\end{equation}
A characteristic tile duration is defined as the bin width:
\begin{equation}
  \Delta\tau_m = T_c / N_\tau,
  \label{eq:dtau}
\end{equation}


%Before performing the analysis of Eq.~\ref{eq:qtransform3}, the data vector $x$ is whitened such that individual samples of the resulting discrete time sequence are statistically independent random variables with zero mean and unit variance.
